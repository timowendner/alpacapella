\section{Datasets}
\subsection{Annotation Dataset}
The annotation dataset contains royalty-free rap acapellas from Looperman\cite{looperman}. The website hosts a variety of free user-uploaded audio content, with royalty-free licenses. The audio files were downloaded manually. The selection focuses on clear vocal content as well as varying tempo, voice and production, to ensure the diversity of the dataset. Beat annotations were created using Sonic Visualiser~\cite{cannam2010}. Each beat and downbeat was manually marked. The unreliable onset information leads to a challenging annotation process.

\textbf{Multi-Annotation Pipeline:} To ensure the best possible quality, a processing pipeline was developed to combine multiple annotations. Each audio file was annotated 3 times independently. 
\begin{itemize}
    \item \textit{IBI Estimation:} For each raw annotation, the time difference between beats was used to estimate the inter beat interval (IBI). The estimate is the mean of the time differences, with outliers removed. The IBI was capped between 0.3 and 1 seconds, representing 200 and 60 BPM, as this covers all the possible timings for rap.
    \item \textit{Missing Beat Interpolation:} If there was a space between two annotations that is larger than what the IBI suggests, then the missing beats were inserted. 
    \item \textit{Local Smoothing:} A smoothing algorithm addressed local timing issues. For each current beat \(t_i\), a window \(W\) of \(\pm 1.5\) inter beat interval was investigated. Within this window, every beat \(t_j\) was moved with a step-size of the IBI-estimation, moving it closest to the current beat.
    \[ t'_i = \frac{1}{N} \sum_{t_j \in W} \left( t_j + IBI \cdot \operatorname{round} \left( \frac{t_i - t_j}{IBI} \right) \right) \]
    \item \textit{Voting Mechanism:} To combine the individual annotations, voting was used. Each time all annotations agree a beat is inside a window of 70ms, the average of the positions is accepted. Otherwise the position is rejected.
    \item \textit{Final Process:} The voting process creates gaps in the timeline. To solve this the missing beat interpolation is used from before. This grid provides the basis for the downbeat insertion. The annotations fix the first downbeat, and the grid extrapolates for every other downbeat. Finally the possible silence parts are filtered out, based on the raw annotations. If the position is within \(\pm 0.5 \) IBI of a raw annotation, it is kept, but discarded otherwise.
\end{itemize}

The multi annotation pipeline has a certain percentage of real vs interpolated beats at the end. A beat is considered real if it originated from the consensus in the voting mechanism. This number is useful for the assessment of the annotation quality. The final dataset excludes annotations with fewer than 50\% real beats. Without the onset information, a possible lag (computer lag, personal bias) cannot be ruled out. The final annotation are shifted \(33ms\) forwards. Both choices are further explored in the Discussions section.

The final dataset showed high diversity. The quality of the rappers varied significantly, ranging from amateur to semi-professional. Different recording qualities, processing steps and languages, including English, German, French, Portuguese have been observed. Some files contained just an isolated hook or verse, while others included the complete song. The dataset is filtered down from 152 files (456 individual annotations) to 110 annotations with accompanying wav files. As shown in Table~\ref{tab:dataset_summary}, the final collection contains over 19,000 manually verified beats and roughly 2.5 hours of audio. The tempo distribution counts 27 slow (\(< 110\) bpm), 35 medium (\(110\) - \(140\) bpm) and 48 fast (\(> 140\) bpm) beats.

\begin{table}[h]
\centering
\begin{tabular}{lrr}
    \toprule
    \textbf{Metric} & \textbf{Raw Dataset} & \textbf{Final Dataset} \\ 
    \midrule
    Total Annotations & 456 & 110 \\
    Total Duration (hh:mm:ss) & 04:10:29 & 02:39:50 \\
    Total Annotated Beats & 74,020 & 19,115 \\ 
    \bottomrule
\end{tabular}
\caption{Dataset Summary and Tempo Distribution.}
\label{tab:dataset_summary}
\end{table}